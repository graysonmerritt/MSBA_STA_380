% Options for packages loaded elsewhere
\PassOptionsToPackage{unicode}{hyperref}
\PassOptionsToPackage{hyphens}{url}
%
\documentclass[
]{article}
\usepackage{amsmath,amssymb}
\usepackage{iftex}
\ifPDFTeX
  \usepackage[T1]{fontenc}
  \usepackage[utf8]{inputenc}
  \usepackage{textcomp} % provide euro and other symbols
\else % if luatex or xetex
  \usepackage{unicode-math} % this also loads fontspec
  \defaultfontfeatures{Scale=MatchLowercase}
  \defaultfontfeatures[\rmfamily]{Ligatures=TeX,Scale=1}
\fi
\usepackage{lmodern}
\ifPDFTeX\else
  % xetex/luatex font selection
\fi
% Use upquote if available, for straight quotes in verbatim environments
\IfFileExists{upquote.sty}{\usepackage{upquote}}{}
\IfFileExists{microtype.sty}{% use microtype if available
  \usepackage[]{microtype}
  \UseMicrotypeSet[protrusion]{basicmath} % disable protrusion for tt fonts
}{}
\makeatletter
\@ifundefined{KOMAClassName}{% if non-KOMA class
  \IfFileExists{parskip.sty}{%
    \usepackage{parskip}
  }{% else
    \setlength{\parindent}{0pt}
    \setlength{\parskip}{6pt plus 2pt minus 1pt}}
}{% if KOMA class
  \KOMAoptions{parskip=half}}
\makeatother
\usepackage{xcolor}
\usepackage[margin=1in]{geometry}
\usepackage{color}
\usepackage{fancyvrb}
\newcommand{\VerbBar}{|}
\newcommand{\VERB}{\Verb[commandchars=\\\{\}]}
\DefineVerbatimEnvironment{Highlighting}{Verbatim}{commandchars=\\\{\}}
% Add ',fontsize=\small' for more characters per line
\usepackage{framed}
\definecolor{shadecolor}{RGB}{248,248,248}
\newenvironment{Shaded}{\begin{snugshade}}{\end{snugshade}}
\newcommand{\AlertTok}[1]{\textcolor[rgb]{0.94,0.16,0.16}{#1}}
\newcommand{\AnnotationTok}[1]{\textcolor[rgb]{0.56,0.35,0.01}{\textbf{\textit{#1}}}}
\newcommand{\AttributeTok}[1]{\textcolor[rgb]{0.13,0.29,0.53}{#1}}
\newcommand{\BaseNTok}[1]{\textcolor[rgb]{0.00,0.00,0.81}{#1}}
\newcommand{\BuiltInTok}[1]{#1}
\newcommand{\CharTok}[1]{\textcolor[rgb]{0.31,0.60,0.02}{#1}}
\newcommand{\CommentTok}[1]{\textcolor[rgb]{0.56,0.35,0.01}{\textit{#1}}}
\newcommand{\CommentVarTok}[1]{\textcolor[rgb]{0.56,0.35,0.01}{\textbf{\textit{#1}}}}
\newcommand{\ConstantTok}[1]{\textcolor[rgb]{0.56,0.35,0.01}{#1}}
\newcommand{\ControlFlowTok}[1]{\textcolor[rgb]{0.13,0.29,0.53}{\textbf{#1}}}
\newcommand{\DataTypeTok}[1]{\textcolor[rgb]{0.13,0.29,0.53}{#1}}
\newcommand{\DecValTok}[1]{\textcolor[rgb]{0.00,0.00,0.81}{#1}}
\newcommand{\DocumentationTok}[1]{\textcolor[rgb]{0.56,0.35,0.01}{\textbf{\textit{#1}}}}
\newcommand{\ErrorTok}[1]{\textcolor[rgb]{0.64,0.00,0.00}{\textbf{#1}}}
\newcommand{\ExtensionTok}[1]{#1}
\newcommand{\FloatTok}[1]{\textcolor[rgb]{0.00,0.00,0.81}{#1}}
\newcommand{\FunctionTok}[1]{\textcolor[rgb]{0.13,0.29,0.53}{\textbf{#1}}}
\newcommand{\ImportTok}[1]{#1}
\newcommand{\InformationTok}[1]{\textcolor[rgb]{0.56,0.35,0.01}{\textbf{\textit{#1}}}}
\newcommand{\KeywordTok}[1]{\textcolor[rgb]{0.13,0.29,0.53}{\textbf{#1}}}
\newcommand{\NormalTok}[1]{#1}
\newcommand{\OperatorTok}[1]{\textcolor[rgb]{0.81,0.36,0.00}{\textbf{#1}}}
\newcommand{\OtherTok}[1]{\textcolor[rgb]{0.56,0.35,0.01}{#1}}
\newcommand{\PreprocessorTok}[1]{\textcolor[rgb]{0.56,0.35,0.01}{\textit{#1}}}
\newcommand{\RegionMarkerTok}[1]{#1}
\newcommand{\SpecialCharTok}[1]{\textcolor[rgb]{0.81,0.36,0.00}{\textbf{#1}}}
\newcommand{\SpecialStringTok}[1]{\textcolor[rgb]{0.31,0.60,0.02}{#1}}
\newcommand{\StringTok}[1]{\textcolor[rgb]{0.31,0.60,0.02}{#1}}
\newcommand{\VariableTok}[1]{\textcolor[rgb]{0.00,0.00,0.00}{#1}}
\newcommand{\VerbatimStringTok}[1]{\textcolor[rgb]{0.31,0.60,0.02}{#1}}
\newcommand{\WarningTok}[1]{\textcolor[rgb]{0.56,0.35,0.01}{\textbf{\textit{#1}}}}
\usepackage{graphicx}
\makeatletter
\def\maxwidth{\ifdim\Gin@nat@width>\linewidth\linewidth\else\Gin@nat@width\fi}
\def\maxheight{\ifdim\Gin@nat@height>\textheight\textheight\else\Gin@nat@height\fi}
\makeatother
% Scale images if necessary, so that they will not overflow the page
% margins by default, and it is still possible to overwrite the defaults
% using explicit options in \includegraphics[width, height, ...]{}
\setkeys{Gin}{width=\maxwidth,height=\maxheight,keepaspectratio}
% Set default figure placement to htbp
\makeatletter
\def\fps@figure{htbp}
\makeatother
\setlength{\emergencystretch}{3em} % prevent overfull lines
\providecommand{\tightlist}{%
  \setlength{\itemsep}{0pt}\setlength{\parskip}{0pt}}
\setcounter{secnumdepth}{-\maxdimen} % remove section numbering
\ifLuaTeX
  \usepackage{selnolig}  % disable illegal ligatures
\fi
\usepackage{bookmark}
\IfFileExists{xurl.sty}{\usepackage{xurl}}{} % add URL line breaks if available
\urlstyle{same}
\hypersetup{
  pdftitle={STA380\_James\_Problems},
  pdfauthor={Grayson Merritt},
  hidelinks,
  pdfcreator={LaTeX via pandoc}}

\title{STA380\_James\_Problems}
\author{Grayson Merritt}
\date{2024-07-31}

\begin{document}
\maketitle

\section{Problem 1}\label{problem-1}

We are looking for P(Yes\textbar TC) P(Yes\textbar RC) = .5
P(No\textbar RC) = .5 P(Yes) = .65 P(No) = .35 P(RC) = .3

The total law of probability states that P(A) = the summation of
P(A\textbar B) * P(B) The P(Yes) comes from only two conditional
probabilities: P(Yes\textbar RC) and P(Yes\textbar TC) So
P(Yes)=P(Yes∣RC)P(RC)+P(Yes∣TC)P(TC) Using some algebra I can arrange
this to (P(Yes) - P(Yes\textbar RC)P(RC)) / P(TC) = P(Yes∣TC) Thus,
P(Yes\textbar TC) is \textbf{71.43\%}

\subsubsection{Part B}\label{part-b}

Sensitivity = P(P\textbar D) = .993 Specificity = P(N\textbar ND) =
.9999 Disease = P(D) = .000025 The question we are solving is: What is
the P(D\textbar P)? Bayes Theorem is P(D\textbar P) =
(P(P\textbar D)\emph{P(D)) / P(P) We have P(P\textbar D) and P(D), so we
need to find P(P). This will require using the rule of total probability
So P(P) = P(P\textbar D) } P(D) + P(P\textbar ND) * P(ND) So we need
P(ND) and P(P\textbar ND) No disease = P(ND) = 1 - P(D) = .999975
P(P\textbar ND) = 1 - P(N\textbar ND) = .0001 (This is the False
Positive case) P(P) = .00012 After calculating all of my needed info and
applying Bayes Theorem, I get that the Probability of a person having
the disease given a positive test is \emph{19.88\%}

\begin{Shaded}
\begin{Highlighting}[]
\NormalTok{p\_yes\_rc }\OtherTok{=}\NormalTok{ .}\DecValTok{5}
\NormalTok{p\_no\_rc }\OtherTok{=}\NormalTok{ .}\DecValTok{5}
\NormalTok{p\_yes }\OtherTok{=}\NormalTok{ .}\DecValTok{65}
\NormalTok{p\_no }\OtherTok{=}\NormalTok{ .}\DecValTok{35}
\NormalTok{p\_rc }\OtherTok{=}\NormalTok{ .}\DecValTok{3}
\NormalTok{p\_tc }\OtherTok{=}\NormalTok{ .}\DecValTok{7}
\NormalTok{p\_yes\_tc }\OtherTok{=}\NormalTok{ (p\_yes}\SpecialCharTok{{-}}\NormalTok{p\_yes\_rc}\SpecialCharTok{*}\NormalTok{p\_rc) }\SpecialCharTok{/}\NormalTok{ p\_tc}
\FunctionTok{print}\NormalTok{(p\_yes\_tc)}
\end{Highlighting}
\end{Shaded}

\begin{verbatim}
## [1] 0.7142857
\end{verbatim}

\begin{Shaded}
\begin{Highlighting}[]
\CommentTok{\# Part B}
\NormalTok{sensitivity }\OtherTok{=}\NormalTok{ .}\DecValTok{993}
\NormalTok{specificity }\OtherTok{=}\NormalTok{ .}\DecValTok{9999}
\NormalTok{disease }\OtherTok{=}\NormalTok{ .}\DecValTok{000025}
\NormalTok{no\_disease }\OtherTok{=} \DecValTok{1}\SpecialCharTok{{-}}\NormalTok{ disease}
\NormalTok{no\_disease}
\end{Highlighting}
\end{Shaded}

\begin{verbatim}
## [1] 0.999975
\end{verbatim}

\begin{Shaded}
\begin{Highlighting}[]
\NormalTok{false\_postive }\OtherTok{=} \DecValTok{1} \SpecialCharTok{{-}}\NormalTok{ specificity}
\NormalTok{false\_postive}
\end{Highlighting}
\end{Shaded}

\begin{verbatim}
## [1] 1e-04
\end{verbatim}

\begin{Shaded}
\begin{Highlighting}[]
\NormalTok{positive }\OtherTok{=}\NormalTok{ sensitivity }\SpecialCharTok{*}\NormalTok{ disease }\SpecialCharTok{+}\NormalTok{ false\_postive }\SpecialCharTok{*}\NormalTok{ no\_disease}
\NormalTok{positive}
\end{Highlighting}
\end{Shaded}

\begin{verbatim}
## [1] 0.0001248225
\end{verbatim}

\begin{Shaded}
\begin{Highlighting}[]
\NormalTok{disease\_given\_positive }\OtherTok{=}\NormalTok{ (sensitivity }\SpecialCharTok{*}\NormalTok{ disease)}\SpecialCharTok{/}\NormalTok{ positive}
\FunctionTok{print}\NormalTok{(disease\_given\_positive)}
\end{Highlighting}
\end{Shaded}

\begin{verbatim}
## [1] 0.1988824
\end{verbatim}

\section{Question 2}\label{question-2}

\subsubsection{Part A}\label{part-a}

This table shows the top ten most popular songs since 1958 based on how
long they were on the billboard 100. Most of these songs were produced
in the last 21 years. I find it interesting that there are no repeats of
performers on this top ten list.

\begin{verbatim}
## -- Attaching core tidyverse packages ------------------------ tidyverse 2.0.0 --
## v dplyr     1.1.4     v readr     2.1.5
## v forcats   1.0.0     v stringr   1.5.1
## v ggplot2   3.5.1     v tibble    3.2.1
## v lubridate 1.9.3     v tidyr     1.3.1
## v purrr     1.0.2     
## -- Conflicts ------------------------------------------ tidyverse_conflicts() --
## x dplyr::filter() masks stats::filter()
## x dplyr::lag()    masks stats::lag()
## i Use the conflicted package (<http://conflicted.r-lib.org/>) to force all conflicts to become errors
## New names:
## Rows: 327895 Columns: 13
## -- Column specification --------------------------------------------------------
## Delimiter: ","
## chr (5): url, week_id, song, performer, song_id
## dbl (8): ...1, week_position, instance, previous_week_position, peak_positio...
## 
## i Use `spec()` to retrieve the full column specification for this data.
## i Specify the column types or set `show_col_types = FALSE` to quiet this message.
## `summarise()` has grouped output by 'performer'. You can override using the `.groups` argument.
\end{verbatim}

\begin{verbatim}
## # A tibble: 29,389 x 3
## # Groups:   performer [10,061]
##    performer                                 song                          count
##    <chr>                                     <chr>                         <int>
##  1 Imagine Dragons                           Radioactive                      87
##  2 AWOLNATION                                Sail                             79
##  3 Jason Mraz                                I'm Yours                        76
##  4 The Weeknd                                Blinding Lights                  76
##  5 LeAnn Rimes                               How Do I Live                    69
##  6 LMFAO Featuring Lauren Bennett & GoonRock Party Rock Anthem                68
##  7 OneRepublic                               Counting Stars                   68
##  8 Adele                                     Rolling In The Deep              65
##  9 Jewel                                     Foolish Games/You Were Meant~    65
## 10 Carrie Underwood                          Before He Cheats                 64
## # i 29,379 more rows
\end{verbatim}

\subsubsection{Part B}\label{part-b-1}

This plot shows the total number of unique songs that charted the
Billboard 100 per year. I guess my parents were correct when they said
music was better in the 80's! The number of unique songs that chart
rapidly declines until around 2002, where more unique songs started to
chart. There was a decline around 2011 followed by rapid unique song
growth.

\begin{Shaded}
\begin{Highlighting}[]
\CommentTok{\#b}
\NormalTok{billboard\_cutoff }\OtherTok{=}\NormalTok{ billboard }\SpecialCharTok{\%\textgreater{}\%} \FunctionTok{filter}\NormalTok{(year }\SpecialCharTok{!=} \DecValTok{1958} \SpecialCharTok{\&}\NormalTok{ year }\SpecialCharTok{!=} \DecValTok{2021}\NormalTok{)}
\NormalTok{table\_with\_counts }\OtherTok{=}\NormalTok{ billboard\_cutoff }\SpecialCharTok{\%\textgreater{}\%} \FunctionTok{group\_by}\NormalTok{(performer,song,year) }\SpecialCharTok{\%\textgreater{}\%} 
  \FunctionTok{summarize}\NormalTok{(}\AttributeTok{total\_count =} \FunctionTok{n}\NormalTok{())}
\end{Highlighting}
\end{Shaded}

\begin{verbatim}
## `summarise()` has grouped output by 'performer', 'song'. You can override using
## the `.groups` argument.
\end{verbatim}

\begin{Shaded}
\begin{Highlighting}[]
\NormalTok{unique\_song\_count }\OtherTok{=}\NormalTok{ table\_with\_counts }\SpecialCharTok{\%\textgreater{}\%} \FunctionTok{group\_by}\NormalTok{(year) }\SpecialCharTok{\%\textgreater{}\%} 
  \FunctionTok{summarize}\NormalTok{(}\AttributeTok{unique\_songs =} \FunctionTok{n}\NormalTok{())}
\FunctionTok{ggplot}\NormalTok{(unique\_song\_count) }\SpecialCharTok{+} \FunctionTok{geom\_line}\NormalTok{(}\FunctionTok{aes}\NormalTok{(}\AttributeTok{x=}\NormalTok{year,}\AttributeTok{y=}\NormalTok{unique\_songs))}
\end{Highlighting}
\end{Shaded}

\includegraphics{STA380_James_Problems_files/figure-latex/Problem 2 Part B-1.pdf}
\#\#\# Part C This plot shows artists who have had 30 songs chart for at
least ten weeks. Elton John has the highest number of songs with 52
songs. I find it interesting that there are a good amount of country
artists filled. I would have thought this list would have been mainly
filled with pop and rock artists

\begin{Shaded}
\begin{Highlighting}[]
\CommentTok{\#C}
\NormalTok{billboard\_ten\_week }\OtherTok{=}\NormalTok{ billboard }\SpecialCharTok{\%\textgreater{}\%} \FunctionTok{group\_by}\NormalTok{(performer,song) }\SpecialCharTok{\%\textgreater{}\%} 
  \FunctionTok{summarize}\NormalTok{(}\AttributeTok{count =} \FunctionTok{n}\NormalTok{()) }\SpecialCharTok{\%\textgreater{}\%}
  \FunctionTok{filter}\NormalTok{(count }\SpecialCharTok{\textgreater{}=}\DecValTok{10}\NormalTok{)}
\end{Highlighting}
\end{Shaded}

\begin{verbatim}
## `summarise()` has grouped output by 'performer'. You can override using the
## `.groups` argument.
\end{verbatim}

\begin{Shaded}
\begin{Highlighting}[]
\NormalTok{billboard\_19\_artists }\OtherTok{=}\NormalTok{ billboard\_ten\_week }\SpecialCharTok{\%\textgreater{}\%} \FunctionTok{group\_by}\NormalTok{(performer) }\SpecialCharTok{\%\textgreater{}\%} 
  \FunctionTok{summarize}\NormalTok{(}\AttributeTok{song\_count =}\FunctionTok{n}\NormalTok{()) }\SpecialCharTok{\%\textgreater{}\%} \FunctionTok{filter}\NormalTok{(song\_count }\SpecialCharTok{\textgreater{}=}\DecValTok{30}\NormalTok{)}
\FunctionTok{ggplot}\NormalTok{(billboard\_19\_artists) }\SpecialCharTok{+} \FunctionTok{geom\_col}\NormalTok{(}\FunctionTok{aes}\NormalTok{(}\AttributeTok{x=}\NormalTok{performer, }\AttributeTok{y=}\NormalTok{song\_count)) }\SpecialCharTok{+} 
  \FunctionTok{coord\_flip}\NormalTok{()}
\end{Highlighting}
\end{Shaded}

\includegraphics{STA380_James_Problems_files/figure-latex/Problem 2 Part C-1.pdf}

Note that the \texttt{echo\ =\ FALSE} parameter was added to the code
chunk to prevent printing of the R code that generated the plot.

\end{document}
