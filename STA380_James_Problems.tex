% Options for packages loaded elsewhere
\PassOptionsToPackage{unicode}{hyperref}
\PassOptionsToPackage{hyphens}{url}
%
\documentclass[
]{article}
\usepackage{amsmath,amssymb}
\usepackage{iftex}
\ifPDFTeX
  \usepackage[T1]{fontenc}
  \usepackage[utf8]{inputenc}
  \usepackage{textcomp} % provide euro and other symbols
\else % if luatex or xetex
  \usepackage{unicode-math} % this also loads fontspec
  \defaultfontfeatures{Scale=MatchLowercase}
  \defaultfontfeatures[\rmfamily]{Ligatures=TeX,Scale=1}
\fi
\usepackage{lmodern}
\ifPDFTeX\else
  % xetex/luatex font selection
\fi
% Use upquote if available, for straight quotes in verbatim environments
\IfFileExists{upquote.sty}{\usepackage{upquote}}{}
\IfFileExists{microtype.sty}{% use microtype if available
  \usepackage[]{microtype}
  \UseMicrotypeSet[protrusion]{basicmath} % disable protrusion for tt fonts
}{}
\makeatletter
\@ifundefined{KOMAClassName}{% if non-KOMA class
  \IfFileExists{parskip.sty}{%
    \usepackage{parskip}
  }{% else
    \setlength{\parindent}{0pt}
    \setlength{\parskip}{6pt plus 2pt minus 1pt}}
}{% if KOMA class
  \KOMAoptions{parskip=half}}
\makeatother
\usepackage{xcolor}
\usepackage[margin=1in]{geometry}
\usepackage{color}
\usepackage{fancyvrb}
\newcommand{\VerbBar}{|}
\newcommand{\VERB}{\Verb[commandchars=\\\{\}]}
\DefineVerbatimEnvironment{Highlighting}{Verbatim}{commandchars=\\\{\}}
% Add ',fontsize=\small' for more characters per line
\usepackage{framed}
\definecolor{shadecolor}{RGB}{248,248,248}
\newenvironment{Shaded}{\begin{snugshade}}{\end{snugshade}}
\newcommand{\AlertTok}[1]{\textcolor[rgb]{0.94,0.16,0.16}{#1}}
\newcommand{\AnnotationTok}[1]{\textcolor[rgb]{0.56,0.35,0.01}{\textbf{\textit{#1}}}}
\newcommand{\AttributeTok}[1]{\textcolor[rgb]{0.13,0.29,0.53}{#1}}
\newcommand{\BaseNTok}[1]{\textcolor[rgb]{0.00,0.00,0.81}{#1}}
\newcommand{\BuiltInTok}[1]{#1}
\newcommand{\CharTok}[1]{\textcolor[rgb]{0.31,0.60,0.02}{#1}}
\newcommand{\CommentTok}[1]{\textcolor[rgb]{0.56,0.35,0.01}{\textit{#1}}}
\newcommand{\CommentVarTok}[1]{\textcolor[rgb]{0.56,0.35,0.01}{\textbf{\textit{#1}}}}
\newcommand{\ConstantTok}[1]{\textcolor[rgb]{0.56,0.35,0.01}{#1}}
\newcommand{\ControlFlowTok}[1]{\textcolor[rgb]{0.13,0.29,0.53}{\textbf{#1}}}
\newcommand{\DataTypeTok}[1]{\textcolor[rgb]{0.13,0.29,0.53}{#1}}
\newcommand{\DecValTok}[1]{\textcolor[rgb]{0.00,0.00,0.81}{#1}}
\newcommand{\DocumentationTok}[1]{\textcolor[rgb]{0.56,0.35,0.01}{\textbf{\textit{#1}}}}
\newcommand{\ErrorTok}[1]{\textcolor[rgb]{0.64,0.00,0.00}{\textbf{#1}}}
\newcommand{\ExtensionTok}[1]{#1}
\newcommand{\FloatTok}[1]{\textcolor[rgb]{0.00,0.00,0.81}{#1}}
\newcommand{\FunctionTok}[1]{\textcolor[rgb]{0.13,0.29,0.53}{\textbf{#1}}}
\newcommand{\ImportTok}[1]{#1}
\newcommand{\InformationTok}[1]{\textcolor[rgb]{0.56,0.35,0.01}{\textbf{\textit{#1}}}}
\newcommand{\KeywordTok}[1]{\textcolor[rgb]{0.13,0.29,0.53}{\textbf{#1}}}
\newcommand{\NormalTok}[1]{#1}
\newcommand{\OperatorTok}[1]{\textcolor[rgb]{0.81,0.36,0.00}{\textbf{#1}}}
\newcommand{\OtherTok}[1]{\textcolor[rgb]{0.56,0.35,0.01}{#1}}
\newcommand{\PreprocessorTok}[1]{\textcolor[rgb]{0.56,0.35,0.01}{\textit{#1}}}
\newcommand{\RegionMarkerTok}[1]{#1}
\newcommand{\SpecialCharTok}[1]{\textcolor[rgb]{0.81,0.36,0.00}{\textbf{#1}}}
\newcommand{\SpecialStringTok}[1]{\textcolor[rgb]{0.31,0.60,0.02}{#1}}
\newcommand{\StringTok}[1]{\textcolor[rgb]{0.31,0.60,0.02}{#1}}
\newcommand{\VariableTok}[1]{\textcolor[rgb]{0.00,0.00,0.00}{#1}}
\newcommand{\VerbatimStringTok}[1]{\textcolor[rgb]{0.31,0.60,0.02}{#1}}
\newcommand{\WarningTok}[1]{\textcolor[rgb]{0.56,0.35,0.01}{\textbf{\textit{#1}}}}
\usepackage{graphicx}
\makeatletter
\def\maxwidth{\ifdim\Gin@nat@width>\linewidth\linewidth\else\Gin@nat@width\fi}
\def\maxheight{\ifdim\Gin@nat@height>\textheight\textheight\else\Gin@nat@height\fi}
\makeatother
% Scale images if necessary, so that they will not overflow the page
% margins by default, and it is still possible to overwrite the defaults
% using explicit options in \includegraphics[width, height, ...]{}
\setkeys{Gin}{width=\maxwidth,height=\maxheight,keepaspectratio}
% Set default figure placement to htbp
\makeatletter
\def\fps@figure{htbp}
\makeatother
\setlength{\emergencystretch}{3em} % prevent overfull lines
\providecommand{\tightlist}{%
  \setlength{\itemsep}{0pt}\setlength{\parskip}{0pt}}
\setcounter{secnumdepth}{-\maxdimen} % remove section numbering
\ifLuaTeX
  \usepackage{selnolig}  % disable illegal ligatures
\fi
\usepackage{bookmark}
\IfFileExists{xurl.sty}{\usepackage{xurl}}{} % add URL line breaks if available
\urlstyle{same}
\hypersetup{
  pdftitle={STA380\_James\_Problems},
  pdfauthor={Grayson Merritt},
  hidelinks,
  pdfcreator={LaTeX via pandoc}}

\title{STA380\_James\_Problems}
\author{Grayson Merritt}
\date{2024-07-31}

\begin{document}
\maketitle

\section{Problem 1}\label{problem-1}

We are looking for P(Yes\textbar TC) P(Yes\textbar RC) = .5
P(No\textbar RC) = .5 P(Yes) = .65 P(No) = .35 P(RC) = .3

The total law of probability states that P(A) = the summation of
P(A\textbar B) * P(B) The P(Yes) comes from only two conditional
probabilities: P(Yes\textbar RC) and P(Yes\textbar TC) So
P(Yes)=P(Yes∣RC)P(RC)+P(Yes∣TC)P(TC) Using some algebra I can arrange
this to (P(Yes) - P(Yes\textbar RC)P(RC)) / P(TC) = P(Yes∣TC) Thus,
P(Yes\textbar TC) is \textbf{71.43\%}

\textbf{Part B} Sensitivity = P(P\textbar D) = .993 Specificity =
P(N\textbar ND) = .9999 Disease = P(D) = .000025 The question we are
solving is: What is the P(D\textbar P)? Bayes Theorem is P(D\textbar P)
= (P(P\textbar D)\emph{P(D)) / P(P) We have P(P\textbar D) and P(D), so
we need to find P(P). This will require using the rule of total
probability So P(P) = P(P\textbar D) } P(D) + P(P\textbar ND) * P(ND) So
we need P(ND) and P(P\textbar ND) No disease = P(ND) = 1 - P(D) =
.999975 P(P\textbar ND) = 1 - P(N\textbar ND) = .0001 (This is the False
Positive case) P(P) = .00012 After calculating all of my needed info and
applying Bayes Theorem, I get that the Probability of a person having
the disease given a positive test is \emph{19.88\%}

\begin{Shaded}
\begin{Highlighting}[]
\NormalTok{p\_yes\_rc }\OtherTok{=}\NormalTok{ .}\DecValTok{5}
\NormalTok{p\_no\_rc }\OtherTok{=}\NormalTok{ .}\DecValTok{5}
\NormalTok{p\_yes }\OtherTok{=}\NormalTok{ .}\DecValTok{65}
\NormalTok{p\_no }\OtherTok{=}\NormalTok{ .}\DecValTok{35}
\NormalTok{p\_rc }\OtherTok{=}\NormalTok{ .}\DecValTok{3}
\NormalTok{p\_tc }\OtherTok{=}\NormalTok{ .}\DecValTok{7}
\NormalTok{p\_yes\_tc }\OtherTok{=}\NormalTok{ (p\_yes}\SpecialCharTok{{-}}\NormalTok{p\_yes\_rc}\SpecialCharTok{*}\NormalTok{p\_rc) }\SpecialCharTok{/}\NormalTok{ p\_tc}
\FunctionTok{print}\NormalTok{(p\_yes\_tc)}
\end{Highlighting}
\end{Shaded}

\begin{verbatim}
## [1] 0.7142857
\end{verbatim}

\begin{Shaded}
\begin{Highlighting}[]
\CommentTok{\# Part B}
\NormalTok{sensitivity }\OtherTok{=}\NormalTok{ .}\DecValTok{993}
\NormalTok{specificity }\OtherTok{=}\NormalTok{ .}\DecValTok{9999}
\NormalTok{disease }\OtherTok{=}\NormalTok{ .}\DecValTok{000025}
\NormalTok{no\_disease }\OtherTok{=} \DecValTok{1}\SpecialCharTok{{-}}\NormalTok{ disease}
\NormalTok{no\_disease}
\end{Highlighting}
\end{Shaded}

\begin{verbatim}
## [1] 0.999975
\end{verbatim}

\begin{Shaded}
\begin{Highlighting}[]
\NormalTok{false\_postive }\OtherTok{=} \DecValTok{1} \SpecialCharTok{{-}}\NormalTok{ specificity}
\NormalTok{false\_postive}
\end{Highlighting}
\end{Shaded}

\begin{verbatim}
## [1] 1e-04
\end{verbatim}

\begin{Shaded}
\begin{Highlighting}[]
\NormalTok{positive }\OtherTok{=}\NormalTok{ sensitivity }\SpecialCharTok{*}\NormalTok{ disease }\SpecialCharTok{+}\NormalTok{ false\_postive }\SpecialCharTok{*}\NormalTok{ no\_disease}
\NormalTok{positive}
\end{Highlighting}
\end{Shaded}

\begin{verbatim}
## [1] 0.0001248225
\end{verbatim}

\begin{Shaded}
\begin{Highlighting}[]
\NormalTok{disease\_given\_positive }\OtherTok{=}\NormalTok{ (sensitivity }\SpecialCharTok{*}\NormalTok{ disease)}\SpecialCharTok{/}\NormalTok{ positive}
\FunctionTok{print}\NormalTok{(disease\_given\_positive)}
\end{Highlighting}
\end{Shaded}

\begin{verbatim}
## [1] 0.1988824
\end{verbatim}

\section{Question 2}\label{question-2}

\textbf{Part A} This table shows the top ten most popular songs since
1958 based on how long they were on the billboard 100. Most of these
songs were produced in the last 21 years. I find it interesting that
there are no repeats of performers on this top ten list.

\begin{verbatim}
## -- Attaching core tidyverse packages ------------------------ tidyverse 2.0.0 --
## v dplyr     1.1.4     v readr     2.1.5
## v forcats   1.0.0     v stringr   1.5.1
## v ggplot2   3.5.1     v tibble    3.2.1
## v lubridate 1.9.3     v tidyr     1.3.1
## v purrr     1.0.2     
## -- Conflicts ------------------------------------------ tidyverse_conflicts() --
## x dplyr::filter() masks stats::filter()
## x dplyr::lag()    masks stats::lag()
## i Use the conflicted package (<http://conflicted.r-lib.org/>) to force all conflicts to become errors
## New names:
## Rows: 327895 Columns: 13
## -- Column specification --------------------------------------------------------
## Delimiter: ","
## chr (5): url, week_id, song, performer, song_id
## dbl (8): ...1, week_position, instance, previous_week_position, peak_positio...
## 
## i Use `spec()` to retrieve the full column specification for this data.
## i Specify the column types or set `show_col_types = FALSE` to quiet this message.
## `summarise()` has grouped output by 'performer'. You can override using the `.groups` argument.
\end{verbatim}

\begin{verbatim}
## # A tibble: 29,389 x 3
## # Groups:   performer [10,061]
##    performer                                 song                          count
##    <chr>                                     <chr>                         <int>
##  1 Imagine Dragons                           Radioactive                      87
##  2 AWOLNATION                                Sail                             79
##  3 Jason Mraz                                I'm Yours                        76
##  4 The Weeknd                                Blinding Lights                  76
##  5 LeAnn Rimes                               How Do I Live                    69
##  6 LMFAO Featuring Lauren Bennett & GoonRock Party Rock Anthem                68
##  7 OneRepublic                               Counting Stars                   68
##  8 Adele                                     Rolling In The Deep              65
##  9 Jewel                                     Foolish Games/You Were Meant~    65
## 10 Carrie Underwood                          Before He Cheats                 64
## # i 29,379 more rows
\end{verbatim}

\textbf{Part B} This plot shows the total number of unique songs that
charted the Billboard 100 per year. I guess my parents were correct when
they said music was better in the 80's! The number of unique songs that
chart peaks at around 1967 and then rapidly declines until around 2002,
where more unique songs started to chart. This could potentially be due
to the rise if iTunes. There was a decline around 2011 followed by rapid
unique song growth.

\begin{Shaded}
\begin{Highlighting}[]
\CommentTok{\#b}
\NormalTok{billboard\_cutoff }\OtherTok{=}\NormalTok{ billboard }\SpecialCharTok{\%\textgreater{}\%} \FunctionTok{filter}\NormalTok{(year }\SpecialCharTok{!=} \DecValTok{1958} \SpecialCharTok{\&}\NormalTok{ year }\SpecialCharTok{!=} \DecValTok{2021}\NormalTok{)}
\NormalTok{table\_with\_counts }\OtherTok{=}\NormalTok{ billboard\_cutoff }\SpecialCharTok{\%\textgreater{}\%} \FunctionTok{group\_by}\NormalTok{(performer,song,year) }\SpecialCharTok{\%\textgreater{}\%} 
  \FunctionTok{summarize}\NormalTok{(}\AttributeTok{total\_count =} \FunctionTok{n}\NormalTok{())}
\end{Highlighting}
\end{Shaded}

\begin{verbatim}
## `summarise()` has grouped output by 'performer', 'song'. You can override using
## the `.groups` argument.
\end{verbatim}

\begin{Shaded}
\begin{Highlighting}[]
\NormalTok{unique\_song\_count }\OtherTok{=}\NormalTok{ table\_with\_counts }\SpecialCharTok{\%\textgreater{}\%} \FunctionTok{group\_by}\NormalTok{(year) }\SpecialCharTok{\%\textgreater{}\%} 
  \FunctionTok{summarize}\NormalTok{(}\AttributeTok{unique\_songs =} \FunctionTok{n}\NormalTok{())}
\FunctionTok{ggplot}\NormalTok{(unique\_song\_count) }\SpecialCharTok{+} \FunctionTok{geom\_line}\NormalTok{(}\FunctionTok{aes}\NormalTok{(}\AttributeTok{x=}\NormalTok{year,}\AttributeTok{y=}\NormalTok{unique\_songs))}
\end{Highlighting}
\end{Shaded}

\includegraphics{STA380_James_Problems_files/figure-latex/Problem 2 Part B-1.pdf}
\textbf{Part C} This plot shows artists who have had 30 songs chart for
at least ten weeks. Elton John has the highest number of songs with 52
songs. I find it interesting that there are a good amount of country
artists filled. I would have thought this list would have been mainly
filled with pop and rock artists

\begin{Shaded}
\begin{Highlighting}[]
\CommentTok{\#C}
\NormalTok{billboard\_ten\_week }\OtherTok{=}\NormalTok{ billboard }\SpecialCharTok{\%\textgreater{}\%} \FunctionTok{group\_by}\NormalTok{(performer,song) }\SpecialCharTok{\%\textgreater{}\%} 
  \FunctionTok{summarize}\NormalTok{(}\AttributeTok{count =} \FunctionTok{n}\NormalTok{()) }\SpecialCharTok{\%\textgreater{}\%}
  \FunctionTok{filter}\NormalTok{(count }\SpecialCharTok{\textgreater{}=}\DecValTok{10}\NormalTok{)}
\end{Highlighting}
\end{Shaded}

\begin{verbatim}
## `summarise()` has grouped output by 'performer'. You can override using the
## `.groups` argument.
\end{verbatim}

\begin{Shaded}
\begin{Highlighting}[]
\NormalTok{billboard\_19\_artists }\OtherTok{=}\NormalTok{ billboard\_ten\_week }\SpecialCharTok{\%\textgreater{}\%} \FunctionTok{group\_by}\NormalTok{(performer) }\SpecialCharTok{\%\textgreater{}\%} 
  \FunctionTok{summarize}\NormalTok{(}\AttributeTok{song\_count =}\FunctionTok{n}\NormalTok{()) }\SpecialCharTok{\%\textgreater{}\%} \FunctionTok{filter}\NormalTok{(song\_count }\SpecialCharTok{\textgreater{}=}\DecValTok{30}\NormalTok{)}
\FunctionTok{ggplot}\NormalTok{(billboard\_19\_artists) }\SpecialCharTok{+} \FunctionTok{geom\_col}\NormalTok{(}\FunctionTok{aes}\NormalTok{(}\AttributeTok{x=}\NormalTok{performer, }\AttributeTok{y=}\NormalTok{song\_count)) }\SpecialCharTok{+} 
  \FunctionTok{coord\_flip}\NormalTok{()}
\end{Highlighting}
\end{Shaded}

\includegraphics{STA380_James_Problems_files/figure-latex/Problem 2 Part C-1.pdf}
\# Problem 3 In order to agree with the stats guru's conclusion that
building a green building makes sense, we first have to do our own
analysis of the data and see what findings we can come up with. We first
took a look at the median rent of green buildings vs non green buildings
and confirmed that on average green buildings earn about \$2.6 more per
sq ft than non green buildings. Our general strategy is to see what
factors could affect rent and see if these factors are over or under
represented in the green buildings.

\begin{Shaded}
\begin{Highlighting}[]
\FunctionTok{library}\NormalTok{(mosaic)}
\end{Highlighting}
\end{Shaded}

\begin{verbatim}
## Registered S3 method overwritten by 'mosaic':
##   method                           from   
##   fortify.SpatialPolygonsDataFrame ggplot2
\end{verbatim}

\begin{verbatim}
## 
## The 'mosaic' package masks several functions from core packages in order to add 
## additional features.  The original behavior of these functions should not be affected by this.
\end{verbatim}

\begin{verbatim}
## 
## Attaching package: 'mosaic'
\end{verbatim}

\begin{verbatim}
## The following object is masked from 'package:Matrix':
## 
##     mean
\end{verbatim}

\begin{verbatim}
## The following objects are masked from 'package:dplyr':
## 
##     count, do, tally
\end{verbatim}

\begin{verbatim}
## The following object is masked from 'package:purrr':
## 
##     cross
\end{verbatim}

\begin{verbatim}
## The following object is masked from 'package:ggplot2':
## 
##     stat
\end{verbatim}

\begin{verbatim}
## The following objects are masked from 'package:stats':
## 
##     binom.test, cor, cor.test, cov, fivenum, IQR, median, prop.test,
##     quantile, sd, t.test, var
\end{verbatim}

\begin{verbatim}
## The following objects are masked from 'package:base':
## 
##     max, mean, min, prod, range, sample, sum
\end{verbatim}

\begin{Shaded}
\begin{Highlighting}[]
\NormalTok{greenbuildings }\OtherTok{=} \FunctionTok{read\_csv}\NormalTok{(}\StringTok{"greenbuildings.csv"}\NormalTok{)}
\end{Highlighting}
\end{Shaded}

\begin{verbatim}
## Rows: 7894 Columns: 23
\end{verbatim}

\begin{verbatim}
## -- Column specification --------------------------------------------------------
## Delimiter: ","
## dbl (23): CS_PropertyID, cluster, size, empl_gr, Rent, leasing_rate, stories...
## 
## i Use `spec()` to retrieve the full column specification for this data.
## i Specify the column types or set `show_col_types = FALSE` to quiet this message.
\end{verbatim}

\begin{Shaded}
\begin{Highlighting}[]
\CommentTok{\# also remove the occupancy rate of less than 10\%}
\NormalTok{greenbuildings }\OtherTok{=}\NormalTok{ greenbuildings }\SpecialCharTok{\%\textgreater{}\%} \FunctionTok{filter}\NormalTok{(}\StringTok{\textquotesingle{}leasing\_rate\textquotesingle{}} \SpecialCharTok{\textgreater{}=} \DecValTok{10}\NormalTok{)}

\FunctionTok{median}\NormalTok{(Rent }\SpecialCharTok{\textasciitilde{}}\NormalTok{ green\_rating, }\AttributeTok{data=}\NormalTok{greenbuildings)}
\end{Highlighting}
\end{Shaded}

\begin{verbatim}
##    0    1 
## 25.0 27.6
\end{verbatim}

\begin{Shaded}
\begin{Highlighting}[]
\FunctionTok{ggplot}\NormalTok{(greenbuildings) }\SpecialCharTok{+} 
  \FunctionTok{geom\_boxplot}\NormalTok{(}\FunctionTok{aes}\NormalTok{(}\AttributeTok{x=}\FunctionTok{factor}\NormalTok{(green\_rating), }\AttributeTok{y=}\NormalTok{Rent)) }\SpecialCharTok{+} 
  \FunctionTok{coord\_flip}\NormalTok{()}
\end{Highlighting}
\end{Shaded}

\includegraphics{STA380_James_Problems_files/figure-latex/problem 3-1.pdf}

We then looked to see if green buildings tended to be more of Class A.
We found that around 80\% of green rated buildings are class A, versus
only 36\% of non green buildings. This may be a potential confounder, as
class A buildings will command more rent.

\begin{Shaded}
\begin{Highlighting}[]
\CommentTok{\# Look at which buildings are "more desirable" (Class A)}
\FunctionTok{xtabs}\NormalTok{(}\SpecialCharTok{\textasciitilde{}}\NormalTok{ class\_a }\SpecialCharTok{+}\NormalTok{ green\_rating, }\AttributeTok{data=}\NormalTok{greenbuildings) }\SpecialCharTok{\%\textgreater{}\%}
  \FunctionTok{prop.table}\NormalTok{(}\AttributeTok{margin=}\DecValTok{2}\NormalTok{)}
\end{Highlighting}
\end{Shaded}

\begin{verbatim}
##        green_rating
## class_a         0         1
##       0 0.6378138 0.2029197
##       1 0.3621862 0.7970803
\end{verbatim}

We plotted Rent vs age and found that an older building commands less
rent. However, the non green buildings are almost 50 years old on
average vs 24 years for green buildings! This could certainly be a
confounder.

\begin{Shaded}
\begin{Highlighting}[]
\CommentTok{\# Look at how age affects rent}
\FunctionTok{mean}\NormalTok{(age }\SpecialCharTok{\textasciitilde{}}\NormalTok{ green\_rating, }\AttributeTok{data=}\NormalTok{greenbuildings)}
\end{Highlighting}
\end{Shaded}

\begin{verbatim}
##        0        1 
## 49.46733 23.84526
\end{verbatim}

\begin{Shaded}
\begin{Highlighting}[]
\FunctionTok{ggplot}\NormalTok{(greenbuildings, }\FunctionTok{aes}\NormalTok{(}\AttributeTok{x =}\NormalTok{ age, }\AttributeTok{y =}\NormalTok{ Rent)) }\SpecialCharTok{+}
  \FunctionTok{geom\_point}\NormalTok{(}\FunctionTok{aes}\NormalTok{(}\AttributeTok{color =} \FunctionTok{as.factor}\NormalTok{(green\_rating)), }\AttributeTok{alpha =} \FloatTok{0.6}\NormalTok{) }\SpecialCharTok{+}
  \FunctionTok{geom\_smooth}\NormalTok{(}\AttributeTok{method =} \StringTok{"loess"}\NormalTok{, }\AttributeTok{se =} \ConstantTok{FALSE}\NormalTok{, }\AttributeTok{color =} \StringTok{"blue"}\NormalTok{) }\SpecialCharTok{+}
  \FunctionTok{labs}\NormalTok{(}\AttributeTok{title =} \StringTok{"Rent vs Age of Buildings"}\NormalTok{,}
       \AttributeTok{x =} \StringTok{"Age of Building (years)"}\NormalTok{,}
       \AttributeTok{y =} \StringTok{"Rent ($ per square foot per year)"}\NormalTok{,}
       \AttributeTok{color =} \StringTok{"Green Rating"}\NormalTok{) }\SpecialCharTok{+}
  \FunctionTok{theme\_minimal}\NormalTok{()}
\end{Highlighting}
\end{Shaded}

\begin{verbatim}
## `geom_smooth()` using formula = 'y ~ x'
\end{verbatim}

\includegraphics{STA380_James_Problems_files/figure-latex/problem 3 age-1.pdf}

Rent is LOWER for renovated buildings. This is not what we expected.
Renovated buildings command less rent by \$3.79! Most green buildings
(around 79\%), however, are NOT renovated! This means that they are
drawing higher prices due to not being renovated.

\begin{Shaded}
\begin{Highlighting}[]
\CommentTok{\# look to see if renovated affects rent}
\FunctionTok{mean}\NormalTok{(Rent }\SpecialCharTok{\textasciitilde{}}\NormalTok{ renovated, }\AttributeTok{data =}\NormalTok{ greenbuildings)}
\end{Highlighting}
\end{Shaded}

\begin{verbatim}
##        0        1 
## 29.85776 26.06570
\end{verbatim}

\begin{Shaded}
\begin{Highlighting}[]
\FunctionTok{xtabs}\NormalTok{(}\SpecialCharTok{\textasciitilde{}}\NormalTok{ renovated }\SpecialCharTok{+}\NormalTok{ green\_rating, }\AttributeTok{data=}\NormalTok{greenbuildings) }\SpecialCharTok{\%\textgreater{}\%}
  \FunctionTok{prop.table}\NormalTok{(}\AttributeTok{margin=}\DecValTok{2}\NormalTok{)}
\end{Highlighting}
\end{Shaded}

\begin{verbatim}
##          green_rating
## renovated         0         1
##         0 0.6046608 0.7868613
##         1 0.3953392 0.2131387
\end{verbatim}

So far we have found 3 plausible explanations for why green buildings
demand higher rent. The class A variable is likely one of the biggest
confounders, so we want to see the median rent within non class A
buildings faceted by green rating. We found that after adjusting for
class A buildings the premium is only \$2.12 now.

\begin{Shaded}
\begin{Highlighting}[]
\CommentTok{\# Look at non class A buildings }
\NormalTok{median\_rent\_non\_class\_a }\OtherTok{\textless{}{-}}\NormalTok{ greenbuildings }\SpecialCharTok{\%\textgreater{}\%}
  \FunctionTok{filter}\NormalTok{(class\_a }\SpecialCharTok{==} \DecValTok{0}\NormalTok{) }\SpecialCharTok{\%\textgreater{}\%}  \CommentTok{\# Exclude Class A buildings}
  \FunctionTok{group\_by}\NormalTok{(green\_rating) }\SpecialCharTok{\%\textgreater{}\%}
  \FunctionTok{summarise}\NormalTok{(}\AttributeTok{median\_rent =} \FunctionTok{median}\NormalTok{(Rent))}

\FunctionTok{print}\NormalTok{(median\_rent\_non\_class\_a)}
\end{Highlighting}
\end{Shaded}

\begin{verbatim}
## # A tibble: 2 x 2
##   green_rating median_rent
##          <dbl>       <dbl>
## 1            0        23.4
## 2            1        25.6
\end{verbatim}

\begin{Shaded}
\begin{Highlighting}[]
\FunctionTok{ggplot}\NormalTok{(median\_rent\_non\_class\_a, }\FunctionTok{aes}\NormalTok{(}\AttributeTok{x =} \FunctionTok{factor}\NormalTok{(green\_rating), }\AttributeTok{y =}\NormalTok{ median\_rent, }\AttributeTok{fill =} \FunctionTok{factor}\NormalTok{(green\_rating))) }\SpecialCharTok{+}
  \FunctionTok{geom\_bar}\NormalTok{(}\AttributeTok{stat =} \StringTok{"identity"}\NormalTok{) }\SpecialCharTok{+}
  \FunctionTok{labs}\NormalTok{(}\AttributeTok{title =} \StringTok{"Median Rent for Non{-}Class A Buildings by Green Rating"}\NormalTok{,}
       \AttributeTok{x =} \StringTok{"Green Rating (0 = Non{-}Green, 1 = Green)"}\NormalTok{,}
       \AttributeTok{y =} \StringTok{"Median Rent ($ per square foot per year)"}\NormalTok{,}
       \AttributeTok{fill =} \StringTok{"Green Rating"}\NormalTok{) }\SpecialCharTok{+}
  \FunctionTok{scale\_fill\_manual}\NormalTok{(}\AttributeTok{values =} \FunctionTok{c}\NormalTok{(}\StringTok{"0"} \OtherTok{=} \StringTok{"grey"}\NormalTok{, }\StringTok{"1"} \OtherTok{=} \StringTok{"green"}\NormalTok{)) }\SpecialCharTok{+}
  \FunctionTok{theme\_minimal}\NormalTok{()}
\end{Highlighting}
\end{Shaded}

\includegraphics{STA380_James_Problems_files/figure-latex/problem 3 trying to match-1.pdf}

We think this trend will continue for most of these confounders. We
think the best way to adjust for these confounders is to use a technique
called matching. Matching relies on a simple principle: compare like
with like. In this example, that means if we have a 25-year-old, Class A
building that is renovated with a green rating, we try to find another
25-year old, Class A renovated building without a green rating to
compare it to. Matching constructs a balanced data set from an
unbalanced one. This matched data can them be compared by their rents to
see if green buildings truly cause a higher premium.

We think the stats guru did not take into account any confounders in his
model. While green building may demand higher rent, we think there any
more variables at play here.

\section{Problem 4}\label{problem-4}

After changing our timestamp to a datetime data type, we plotted the
average boardings by a few different variables.

\begin{Shaded}
\begin{Highlighting}[]
\NormalTok{file\_path }\OtherTok{\textless{}{-}} \StringTok{"capmetro\_UT.csv"}
\NormalTok{capmetro\_UT }\OtherTok{\textless{}{-}} \FunctionTok{read.csv}\NormalTok{(file\_path)}

\CommentTok{\# Convert timestamp to datetime}
\NormalTok{capmetro\_UT}\SpecialCharTok{$}\NormalTok{timestamp }\OtherTok{\textless{}{-}} \FunctionTok{ymd\_hms}\NormalTok{(capmetro\_UT}\SpecialCharTok{$}\NormalTok{timestamp)}

\CommentTok{\# average boardings by hour of the day}
\NormalTok{hour\_summary }\OtherTok{\textless{}{-}}\NormalTok{ capmetro\_UT }\SpecialCharTok{\%\textgreater{}\%}
  \FunctionTok{group\_by}\NormalTok{(hour\_of\_day) }\SpecialCharTok{\%\textgreater{}\%}
  \FunctionTok{summarize}\NormalTok{(}\AttributeTok{mean\_boardings =} \FunctionTok{mean}\NormalTok{(boarding))}

\FunctionTok{ggplot}\NormalTok{(hour\_summary) }\SpecialCharTok{+} 
  \FunctionTok{geom\_line}\NormalTok{(}\FunctionTok{aes}\NormalTok{(}\AttributeTok{x =}\NormalTok{ hour\_of\_day, }\AttributeTok{y =}\NormalTok{ mean\_boardings)) }\SpecialCharTok{+}
  \FunctionTok{labs}\NormalTok{(}\AttributeTok{title =} \StringTok{"Average Boardings by Hour of the Day"}\NormalTok{,}
       \AttributeTok{x =} \StringTok{"Hour of the Day"}\NormalTok{,}
       \AttributeTok{y =} \StringTok{"Average Number of Boardings"}\NormalTok{) }\SpecialCharTok{+}
  \FunctionTok{theme\_minimal}\NormalTok{()}
\end{Highlighting}
\end{Shaded}

\includegraphics{STA380_James_Problems_files/figure-latex/problem 4-1.pdf}
The relationship between boardings and hour of the day is obviously
nonlinear. There is a peak in the late afternoon, followed by a lull
overnight and in the early morning, when fewer people ride the bus.

\begin{Shaded}
\begin{Highlighting}[]
\CommentTok{\# Convert day\_of\_week to a factor with levels in the correct order}
\NormalTok{capmetro\_UT}\SpecialCharTok{$}\NormalTok{day\_of\_week }\OtherTok{\textless{}{-}} \FunctionTok{factor}\NormalTok{(capmetro\_UT}\SpecialCharTok{$}\NormalTok{day\_of\_week, }
                                  \AttributeTok{levels =} \FunctionTok{c}\NormalTok{(}\StringTok{"Mon"}\NormalTok{, }\StringTok{"Tue"}\NormalTok{, }\StringTok{"Wed"}\NormalTok{, }\StringTok{"Thu"}\NormalTok{, }\StringTok{"Fri"}\NormalTok{, }\StringTok{"Sat"}\NormalTok{, }\StringTok{"Sun"}\NormalTok{))}

\CommentTok{\# Summary of mean boardings and alightings by day of the week}
\NormalTok{day\_summary }\OtherTok{\textless{}{-}}\NormalTok{ capmetro\_UT }\SpecialCharTok{\%\textgreater{}\%}
  \FunctionTok{group\_by}\NormalTok{(day\_of\_week) }\SpecialCharTok{\%\textgreater{}\%}
  \FunctionTok{summarise}\NormalTok{(}\AttributeTok{mean\_boardings =} \FunctionTok{mean}\NormalTok{(boarding), }\AttributeTok{mean\_alightings =} \FunctionTok{mean}\NormalTok{(alighting)) }\SpecialCharTok{\%\textgreater{}\%}
  \FunctionTok{pivot\_longer}\NormalTok{(}\AttributeTok{cols =} \FunctionTok{c}\NormalTok{(mean\_boardings, mean\_alightings), }\AttributeTok{names\_to =} \StringTok{"type"}\NormalTok{, }\AttributeTok{values\_to =} \StringTok{"count"}\NormalTok{)}

\CommentTok{\# Plot average boardings and alightings by day of the week}
\FunctionTok{ggplot}\NormalTok{(day\_summary, }\FunctionTok{aes}\NormalTok{(}\AttributeTok{x =}\NormalTok{ day\_of\_week, }\AttributeTok{y =}\NormalTok{ count, }\AttributeTok{fill =}\NormalTok{ type)) }\SpecialCharTok{+} 
  \FunctionTok{geom\_bar}\NormalTok{(}\AttributeTok{stat =} \StringTok{"identity"}\NormalTok{, }\AttributeTok{position =} \StringTok{"dodge"}\NormalTok{) }\SpecialCharTok{+}
  \FunctionTok{labs}\NormalTok{(}\AttributeTok{title =} \StringTok{"Average Boardings and Alightings by Day of the Week"}\NormalTok{,}
       \AttributeTok{x =} \StringTok{"Day of the Week"}\NormalTok{,}
       \AttributeTok{y =} \StringTok{"Average Count"}\NormalTok{,}
       \AttributeTok{fill =} \StringTok{"Type"}\NormalTok{) }\SpecialCharTok{+}
  \FunctionTok{theme\_minimal}\NormalTok{()}
\end{Highlighting}
\end{Shaded}

\includegraphics{STA380_James_Problems_files/figure-latex/prob 4 2-1.pdf}
There are more boardings than alightings for every day. It would make
sense that boardings would be more accurately tracked than alightings,
so we hypothesize that there is a proportion of alightings that is not
captured each day. Another hypthesis is that this is data only for UT
bus stops. That would mean that more people leave campus on a metro than
arrive to campus on a metro. Saturday and Sunday have less boardings and
alightings due to classes not being on those days. Boardings and
alightings peak on Tuesday and then tail off towards the end of the
week. There are less classes on Friday typically, so this drop makes
sense.

\begin{Shaded}
\begin{Highlighting}[]
\NormalTok{coldest\_temperatures }\OtherTok{\textless{}{-}}\NormalTok{ capmetro\_UT }\SpecialCharTok{\%\textgreater{}\%}
  \FunctionTok{arrange}\NormalTok{(temperature) }\SpecialCharTok{\%\textgreater{}\%}
  \FunctionTok{select}\NormalTok{(timestamp, temperature) }\SpecialCharTok{\%\textgreater{}\%}
  \FunctionTok{head}\NormalTok{()}

\CommentTok{\# Print the coldest few values}
\FunctionTok{print}\NormalTok{(coldest\_temperatures)}
\end{Highlighting}
\end{Shaded}

\begin{verbatim}
##             timestamp temperature
## 1 2018-11-14 06:00:00       29.18
## 2 2018-11-14 06:15:00       29.18
## 3 2018-11-14 06:30:00       29.18
## 4 2018-11-14 06:45:00       29.18
## 5 2018-11-14 07:00:00       29.27
## 6 2018-11-14 07:15:00       29.27
\end{verbatim}

\begin{Shaded}
\begin{Highlighting}[]
\FunctionTok{ggplot}\NormalTok{() }\SpecialCharTok{+}
  \FunctionTok{geom\_point}\NormalTok{(}\AttributeTok{data =}\NormalTok{ capmetro\_UT, }\FunctionTok{aes}\NormalTok{(}\AttributeTok{x =}\NormalTok{ temperature, }\AttributeTok{y =}\NormalTok{ boarding, }\AttributeTok{color =} \StringTok{"Boarding"}\NormalTok{), }\AttributeTok{alpha =} \FloatTok{0.5}\NormalTok{) }\SpecialCharTok{+}
  \FunctionTok{geom\_point}\NormalTok{(}\AttributeTok{data =}\NormalTok{ capmetro\_UT, }\FunctionTok{aes}\NormalTok{(}\AttributeTok{x =}\NormalTok{ temperature, }\AttributeTok{y =}\NormalTok{ alighting, }\AttributeTok{color =} \StringTok{"Alighting"}\NormalTok{), }\AttributeTok{alpha =} \FloatTok{0.5}\NormalTok{) }\SpecialCharTok{+}
  \FunctionTok{geom\_smooth}\NormalTok{(}\AttributeTok{data =}\NormalTok{ capmetro\_UT, }\FunctionTok{aes}\NormalTok{(}\AttributeTok{x =}\NormalTok{ temperature, }\AttributeTok{y =}\NormalTok{ boarding, }\AttributeTok{color =} \StringTok{"Boarding"}\NormalTok{), }\AttributeTok{method =} \StringTok{"lm"}\NormalTok{, }\AttributeTok{se =} \ConstantTok{FALSE}\NormalTok{) }\SpecialCharTok{+}
  \FunctionTok{geom\_smooth}\NormalTok{(}\AttributeTok{data =}\NormalTok{ capmetro\_UT, }\FunctionTok{aes}\NormalTok{(}\AttributeTok{x =}\NormalTok{ temperature, }\AttributeTok{y =}\NormalTok{ alighting, }\AttributeTok{color =} \StringTok{"Alighting"}\NormalTok{), }\AttributeTok{method =} \StringTok{"lm"}\NormalTok{, }\AttributeTok{se =} \ConstantTok{FALSE}\NormalTok{) }\SpecialCharTok{+}
  \FunctionTok{labs}\NormalTok{(}\AttributeTok{title =} \StringTok{"Ridership vs. Temperature"}\NormalTok{,}
       \AttributeTok{x =} \StringTok{"Temperature (F)"}\NormalTok{,}
       \AttributeTok{y =} \StringTok{"Count"}\NormalTok{,}
       \AttributeTok{color =} \StringTok{"Type"}\NormalTok{) }\SpecialCharTok{+}
  \FunctionTok{scale\_color\_manual}\NormalTok{(}\AttributeTok{values =} \FunctionTok{c}\NormalTok{(}\StringTok{"Boarding"} \OtherTok{=} \StringTok{"blue"}\NormalTok{, }\StringTok{"Alighting"} \OtherTok{=} \StringTok{"red"}\NormalTok{)) }\SpecialCharTok{+}
  \FunctionTok{theme\_minimal}\NormalTok{()}
\end{Highlighting}
\end{Shaded}

\begin{verbatim}
## `geom_smooth()` using formula = 'y ~ x'
## `geom_smooth()` using formula = 'y ~ x'
\end{verbatim}

\includegraphics{STA380_James_Problems_files/figure-latex/problem 4 3-1.pdf}
This plot is very messy. Of note is that the coldest temperature that
Fall 2018 Semester is around 29 degrees. It appears that there is a
positive relationship between boardings and temperature but a negative
relationship between temperature and alighting! The trendlines are
likely influenced by outliers though.

\begin{Shaded}
\begin{Highlighting}[]
\CommentTok{\# Convert month to a factor with levels in the correct order}
\NormalTok{capmetro\_UT}\SpecialCharTok{$}\NormalTok{month }\OtherTok{\textless{}{-}} \FunctionTok{factor}\NormalTok{(capmetro\_UT}\SpecialCharTok{$}\NormalTok{month, }
                            \AttributeTok{levels =} \FunctionTok{c}\NormalTok{(}\StringTok{"Sep"}\NormalTok{, }\StringTok{"Oct"}\NormalTok{, }\StringTok{"Nov"}\NormalTok{))}

\CommentTok{\# Summary of mean boardings and alightings by month}
\NormalTok{month\_summary }\OtherTok{\textless{}{-}}\NormalTok{ capmetro\_UT }\SpecialCharTok{\%\textgreater{}\%}
  \FunctionTok{group\_by}\NormalTok{(month) }\SpecialCharTok{\%\textgreater{}\%}
  \FunctionTok{summarise}\NormalTok{(}\AttributeTok{mean\_boardings =} \FunctionTok{mean}\NormalTok{(boarding, }\AttributeTok{na.rm =} \ConstantTok{TRUE}\NormalTok{), }
            \AttributeTok{mean\_alightings =} \FunctionTok{mean}\NormalTok{(alighting, }\AttributeTok{na.rm =} \ConstantTok{TRUE}\NormalTok{)) }\SpecialCharTok{\%\textgreater{}\%}
  \FunctionTok{pivot\_longer}\NormalTok{(}\AttributeTok{cols =} \FunctionTok{c}\NormalTok{(mean\_boardings, mean\_alightings), }\AttributeTok{names\_to =} \StringTok{"type"}\NormalTok{, }\AttributeTok{values\_to =} \StringTok{"count"}\NormalTok{)}

\CommentTok{\# Plot average boardings and alightings by month}
\FunctionTok{ggplot}\NormalTok{(month\_summary, }\FunctionTok{aes}\NormalTok{(}\AttributeTok{x =}\NormalTok{ month, }\AttributeTok{y =}\NormalTok{ count, }\AttributeTok{fill =}\NormalTok{ type)) }\SpecialCharTok{+} 
  \FunctionTok{geom\_bar}\NormalTok{(}\AttributeTok{stat =} \StringTok{"identity"}\NormalTok{, }\AttributeTok{position =} \StringTok{"dodge"}\NormalTok{) }\SpecialCharTok{+}
  \FunctionTok{labs}\NormalTok{(}\AttributeTok{title =} \StringTok{"Average Boardings and Alightings by Month"}\NormalTok{,}
       \AttributeTok{x =} \StringTok{"Month"}\NormalTok{,}
       \AttributeTok{y =} \StringTok{"Average Count"}\NormalTok{,}
       \AttributeTok{fill =} \StringTok{"Type"}\NormalTok{) }\SpecialCharTok{+}
    \FunctionTok{theme\_minimal}\NormalTok{()}
\end{Highlighting}
\end{Shaded}

\includegraphics{STA380_James_Problems_files/figure-latex/prob 4 4-1.pdf}
There are more boardings and alightings in October than September. This
is likely due to there being no holidays in October, making it a busier
month. Football games are in full force and Halloween is a big deal.
Conversely, November has less people on the metro. We think this is due
to school burnout causing less people to go to school plus the
Thanksgiving break that affects at least a week of data.

\begin{Shaded}
\begin{Highlighting}[]
\CommentTok{\# ridership by weekend status}
\NormalTok{weekend\_summary }\OtherTok{\textless{}{-}}\NormalTok{ capmetro\_UT }\SpecialCharTok{\%\textgreater{}\%}
  \FunctionTok{group\_by}\NormalTok{(weekend, hour\_of\_day) }\SpecialCharTok{\%\textgreater{}\%}
  \FunctionTok{summarise}\NormalTok{(}\AttributeTok{mean\_boardings =} \FunctionTok{mean}\NormalTok{(boarding), }\AttributeTok{mean\_alightings =} \FunctionTok{mean}\NormalTok{(alighting))}
\end{Highlighting}
\end{Shaded}

\begin{verbatim}
## `summarise()` has grouped output by 'weekend'. You can override using the
## `.groups` argument.
\end{verbatim}

\begin{Shaded}
\begin{Highlighting}[]
\FunctionTok{ggplot}\NormalTok{(weekend\_summary) }\SpecialCharTok{+} 
  \FunctionTok{geom\_line}\NormalTok{(}\FunctionTok{aes}\NormalTok{(}\AttributeTok{x =}\NormalTok{ hour\_of\_day, }\AttributeTok{y =}\NormalTok{ mean\_boardings, }\AttributeTok{color =} \StringTok{"Boardings"}\NormalTok{), }\AttributeTok{size =} \DecValTok{1}\NormalTok{) }\SpecialCharTok{+}
  \FunctionTok{geom\_line}\NormalTok{(}\FunctionTok{aes}\NormalTok{(}\AttributeTok{x =}\NormalTok{ hour\_of\_day, }\AttributeTok{y =}\NormalTok{ mean\_alightings, }\AttributeTok{color =} \StringTok{"Alightings"}\NormalTok{), }\AttributeTok{size =} \DecValTok{1}\NormalTok{) }\SpecialCharTok{+}
  \FunctionTok{facet\_wrap}\NormalTok{(}\SpecialCharTok{\textasciitilde{}}\NormalTok{weekend) }\SpecialCharTok{+}
  \FunctionTok{labs}\NormalTok{(}\AttributeTok{title =} \StringTok{"Ridership by Hour: Weekday vs. Weekend"}\NormalTok{,}
       \AttributeTok{x =} \StringTok{"Hour of the Day"}\NormalTok{,}
       \AttributeTok{y =} \StringTok{"Average Count"}\NormalTok{,}
       \AttributeTok{color =} \StringTok{"Type"}\NormalTok{) }\SpecialCharTok{+}
  \FunctionTok{theme\_minimal}\NormalTok{()}
\end{Highlighting}
\end{Shaded}

\begin{verbatim}
## Warning: Using `size` aesthetic for lines was deprecated in ggplot2 3.4.0.
## i Please use `linewidth` instead.
## This warning is displayed once every 8 hours.
## Call `lifecycle::last_lifecycle_warnings()` to see where this warning was
## generated.
\end{verbatim}

\includegraphics{STA380_James_Problems_files/figure-latex/problem 4 5-1.pdf}
The graphs between weekday and weekend are strikingly different!
Weekends see a lot less traffic as classes are not in session. Weekdays
have a peak alighting at around 8-9AM as students and faculty head to
class. There is more boardings beyond 12PM with a peak at around 6PM as
students and faculty leave campus for the day.

\section{Problem 5}\label{problem-5}

First we had to load our data and engineer it into a usable format for
our clustering algorithms. We narrowed the data down to only the 11
chemical properties and then scaled the data while also removing the
data.

\begin{Shaded}
\begin{Highlighting}[]
\FunctionTok{library}\NormalTok{(Rtsne)}

\CommentTok{\# Keep just the chemicals, scale the data, and then remove duplicates}
\FunctionTok{set.seed}\NormalTok{(}\DecValTok{19}\NormalTok{)}
\NormalTok{wine }\OtherTok{=} \FunctionTok{read.csv}\NormalTok{(}\StringTok{\textquotesingle{}wine.csv\textquotesingle{}}\NormalTok{)}
\NormalTok{wine }\OtherTok{=} \FunctionTok{unique}\NormalTok{(wine)}
\NormalTok{wineSubset }\OtherTok{=} \FunctionTok{select}\NormalTok{(wine, }\SpecialCharTok{{-}}\NormalTok{color, }\SpecialCharTok{{-}}\NormalTok{quality)}
\NormalTok{wineSubsetScaled }\OtherTok{=} \FunctionTok{scale}\NormalTok{(wineSubset)}
\NormalTok{duplicate\_rows }\OtherTok{=} \FunctionTok{duplicated}\NormalTok{(wineSubsetScaled) }\SpecialCharTok{|} \FunctionTok{duplicated}\NormalTok{(wineSubsetScaled, }\AttributeTok{fromLast =} \ConstantTok{TRUE}\NormalTok{)}
\NormalTok{wineSubsetScaled }\OtherTok{=}\NormalTok{ wineSubsetScaled[}\SpecialCharTok{!}\NormalTok{duplicate\_rows, ]}
\NormalTok{wine }\OtherTok{=}\NormalTok{ wine[}\SpecialCharTok{!}\NormalTok{duplicate\_rows, ]}
\end{Highlighting}
\end{Shaded}

Then we run PCA on the data. About half the data can be captured in two
dimensions. The dataframe shows how much data is captured by each PCA
component. The graph shows that PCA does a decent job separating the
wines into color based on the chemical properties. The clusters are in
close vicinity after reducing the dimensions from 11 to 2. There is some
overlap in the middle between the two clusters.

\begin{Shaded}
\begin{Highlighting}[]
\CommentTok{\# scale allows us to make sure each variable contributes equally }
\NormalTok{pcaResult }\OtherTok{=} \FunctionTok{prcomp}\NormalTok{(wineSubsetScaled, }\AttributeTok{scale =} \ConstantTok{TRUE}\NormalTok{)}

\CommentTok{\# get the first and second principal component }
\NormalTok{pcaDataColor }\OtherTok{=} \FunctionTok{data.frame}\NormalTok{(}\AttributeTok{PC1 =}\NormalTok{ pcaResult}\SpecialCharTok{$}\NormalTok{x[, }\DecValTok{1}\NormalTok{], }\AttributeTok{PC2 =}\NormalTok{ pcaResult}\SpecialCharTok{$}\NormalTok{x[, }\DecValTok{2}\NormalTok{], }\AttributeTok{color =}\NormalTok{ wine}\SpecialCharTok{$}\NormalTok{color)}

\FunctionTok{ggplot}\NormalTok{(pcaDataColor, }\FunctionTok{aes}\NormalTok{(}\AttributeTok{x =}\NormalTok{ PC1, }\AttributeTok{y =}\NormalTok{ PC2, }\AttributeTok{color =}\NormalTok{ color)) }\SpecialCharTok{+}
  \FunctionTok{geom\_point}\NormalTok{() }\SpecialCharTok{+}
  \FunctionTok{labs}\NormalTok{(}\AttributeTok{title =} \StringTok{"PCA Visualization by Color"}\NormalTok{, }\AttributeTok{x =} \StringTok{"Principal Component 1"}\NormalTok{, }\AttributeTok{y =} \StringTok{"Principal Component 2"}\NormalTok{) }\SpecialCharTok{+}
  \FunctionTok{scale\_color\_manual}\NormalTok{(}\AttributeTok{values =} \FunctionTok{c}\NormalTok{(}\StringTok{"white"} \OtherTok{=} \StringTok{"blue"}\NormalTok{, }\StringTok{"red"} \OtherTok{=} \StringTok{"red"}\NormalTok{))}\SpecialCharTok{+}
  \FunctionTok{theme\_minimal}\NormalTok{()}
\end{Highlighting}
\end{Shaded}

\includegraphics{STA380_James_Problems_files/figure-latex/prob 5 PCA-1.pdf}
We then used K means clustering to try and see how accurate PCA is. The
clustering shows a majority of reds in one cluster and a majority of
whites in the other. This is good news. The accuracy was 98.16\%.

\begin{Shaded}
\begin{Highlighting}[]
\CommentTok{\# Perform K{-}means clustering on the PCA{-}transformed data}
\NormalTok{kmeansPCAResultColor }\OtherTok{=} \FunctionTok{kmeans}\NormalTok{(pcaDataColor[, }\DecValTok{1}\SpecialCharTok{:}\DecValTok{2}\NormalTok{], }\AttributeTok{centers =} \DecValTok{2}\NormalTok{, }\AttributeTok{nstart =} \DecValTok{20}\NormalTok{)}
\NormalTok{pcaDataColor}\SpecialCharTok{$}\NormalTok{clusterColor }\OtherTok{=} \FunctionTok{as.factor}\NormalTok{(kmeansPCAResultColor}\SpecialCharTok{$}\NormalTok{cluster)}

\CommentTok{\# Create a confusion matrix to compare the actual color with the cluster assignments}
\NormalTok{confusion\_matrix\_pca }\OtherTok{\textless{}{-}} \FunctionTok{table}\NormalTok{(pcaDataColor}\SpecialCharTok{$}\NormalTok{color, pcaDataColor}\SpecialCharTok{$}\NormalTok{clusterColor)}
\FunctionTok{print}\NormalTok{(confusion\_matrix\_pca)}
\end{Highlighting}
\end{Shaded}

\begin{verbatim}
##        
##            1    2
##   red   1329   28
##   white   70 3889
\end{verbatim}

\begin{Shaded}
\begin{Highlighting}[]
\CommentTok{\# Calculate the accuracy}
\NormalTok{correct\_labels\_pca }\OtherTok{\textless{}{-}} \FunctionTok{sum}\NormalTok{(}\FunctionTok{diag}\NormalTok{(confusion\_matrix\_pca))  }\CommentTok{\# Sum of diagonal elements}
\NormalTok{total\_labels\_pca }\OtherTok{\textless{}{-}} \FunctionTok{sum}\NormalTok{(confusion\_matrix\_pca)  }\CommentTok{\# Sum of all elements}
\NormalTok{accuracy\_pca }\OtherTok{\textless{}{-}}\NormalTok{ correct\_labels\_pca }\SpecialCharTok{/}\NormalTok{ total\_labels\_pca  }\CommentTok{\# Accuracy}

\CommentTok{\# Print the accuracy}
\FunctionTok{print}\NormalTok{(}\FunctionTok{paste}\NormalTok{(}\StringTok{"Accuracy: "}\NormalTok{, }\FunctionTok{round}\NormalTok{(accuracy\_pca, }\DecValTok{4}\NormalTok{)))}
\end{Highlighting}
\end{Shaded}

\begin{verbatim}
## [1] "Accuracy:  0.9816"
\end{verbatim}

\begin{Shaded}
\begin{Highlighting}[]
\CommentTok{\# Visualization of the PCA with clusters}
\FunctionTok{ggplot}\NormalTok{(pcaDataColor, }\FunctionTok{aes}\NormalTok{(}\AttributeTok{x =}\NormalTok{ PC1, }\AttributeTok{y =}\NormalTok{ PC2, }\AttributeTok{color =}\NormalTok{ clusterColor)) }\SpecialCharTok{+}
  \FunctionTok{geom\_point}\NormalTok{() }\SpecialCharTok{+}
  \FunctionTok{labs}\NormalTok{(}\AttributeTok{title =} \StringTok{"PCA Visualization with K{-}means Clusters"}\NormalTok{, }\AttributeTok{x =} \StringTok{"Principal Component 1"}\NormalTok{, }\AttributeTok{y =} \StringTok{"Principal Component 2"}\NormalTok{) }\SpecialCharTok{+}
  \FunctionTok{scale\_color\_manual}\NormalTok{(}\AttributeTok{values =} \FunctionTok{c}\NormalTok{(}\StringTok{"1"} \OtherTok{=} \StringTok{"blue"}\NormalTok{, }\StringTok{"2"} \OtherTok{=} \StringTok{"red"}\NormalTok{)) }\SpecialCharTok{+}
  \FunctionTok{theme\_minimal}\NormalTok{()}
\end{Highlighting}
\end{Shaded}

\includegraphics{STA380_James_Problems_files/figure-latex/prob 5 kmeans on PCA-1.pdf}
We then tried to see if PCA can predict the wine quality correctly.
Judging by our graph, it does a pretty poor job on both clusters. Wine
quality is likely much more subjective than if the wine is red vs white
based on its chemical components!

\begin{Shaded}
\begin{Highlighting}[]
\CommentTok{\# Add quality to the data frame for visualization}
\NormalTok{pcaDataQuality }\OtherTok{=} \FunctionTok{data.frame}\NormalTok{(}\AttributeTok{PC1 =}\NormalTok{ pcaResult}\SpecialCharTok{$}\NormalTok{x[, }\DecValTok{1}\NormalTok{], }\AttributeTok{PC2 =}\NormalTok{ pcaResult}\SpecialCharTok{$}\NormalTok{x[, }\DecValTok{2}\NormalTok{], }
                            \AttributeTok{quality =}\NormalTok{ wine}\SpecialCharTok{$}\NormalTok{quality, }\AttributeTok{cluster =}\NormalTok{ pcaDataColor}\SpecialCharTok{$}\NormalTok{clusterColor)}

\FunctionTok{ggplot}\NormalTok{(pcaDataQuality, }\FunctionTok{aes}\NormalTok{(}\AttributeTok{x =}\NormalTok{ PC1, }\AttributeTok{y =}\NormalTok{ PC2, }\AttributeTok{color =} \FunctionTok{as.factor}\NormalTok{(quality))) }\SpecialCharTok{+}
  \FunctionTok{geom\_point}\NormalTok{() }\SpecialCharTok{+}
  \FunctionTok{facet\_wrap}\NormalTok{(}\SpecialCharTok{\textasciitilde{}}\NormalTok{ cluster) }\SpecialCharTok{+}
  \FunctionTok{labs}\NormalTok{(}\AttributeTok{title =} \StringTok{"PCA Visualization by Quality and Cluster"}\NormalTok{, }\AttributeTok{x =} \StringTok{"Principal Component 1"}\NormalTok{, }\AttributeTok{y =} \StringTok{"Principal Component 2"}\NormalTok{) }\SpecialCharTok{+}
  \FunctionTok{theme\_minimal}\NormalTok{()}
\end{Highlighting}
\end{Shaded}

\includegraphics{STA380_James_Problems_files/figure-latex/prob 5 PCA on quailty-1.pdf}

Next we ran tSNE. Perhaps the linear PCA summary is not very good and is
misleading us, and tSNE can help us out here with its nonlinear
dimensonality reduction. We decided to do this in R to keep this problem
all on the same file (hopefully this is a good idea!). There is more of
a defined boundary in TSNE than PCA. There are a few more whites that
are miscalssified as red than reds misclassified as white.

\begin{Shaded}
\begin{Highlighting}[]
\NormalTok{tsneResult }\OtherTok{=} \FunctionTok{Rtsne}\NormalTok{(wineSubsetScaled, }\AttributeTok{dims =} \DecValTok{2}\NormalTok{)}
\NormalTok{tsneData }\OtherTok{=} \FunctionTok{data.frame}\NormalTok{(}\AttributeTok{TSNE\_1 =}\NormalTok{ tsneResult}\SpecialCharTok{$}\NormalTok{Y[, }\DecValTok{1}\NormalTok{], }\AttributeTok{TSNE\_2 =}\NormalTok{ tsneResult}\SpecialCharTok{$}\NormalTok{Y[, }\DecValTok{2}\NormalTok{], }\AttributeTok{color =}\NormalTok{ wine}\SpecialCharTok{$}\NormalTok{color)}

\FunctionTok{ggplot}\NormalTok{(tsneData, }\FunctionTok{aes}\NormalTok{(}\AttributeTok{x =}\NormalTok{ TSNE\_1, }\AttributeTok{y =}\NormalTok{ TSNE\_2, }\AttributeTok{color =}\NormalTok{ color)) }\SpecialCharTok{+}
  \FunctionTok{geom\_point}\NormalTok{() }\SpecialCharTok{+}
  \FunctionTok{labs}\NormalTok{(}\AttributeTok{title =} \StringTok{"t{-}SNE Visualization by Color"}\NormalTok{, }\AttributeTok{x =} \StringTok{"t{-}SNE Dimension 1"}\NormalTok{, }\AttributeTok{y =} \StringTok{"t{-}SNE Dimension 2"}\NormalTok{) }\SpecialCharTok{+}
  \FunctionTok{scale\_color\_manual}\NormalTok{(}\AttributeTok{values =} \FunctionTok{c}\NormalTok{(}\StringTok{"white"} \OtherTok{=} \StringTok{"blue"}\NormalTok{, }\StringTok{"red"} \OtherTok{=} \StringTok{"red"}\NormalTok{))}\SpecialCharTok{+}
  \FunctionTok{theme\_minimal}\NormalTok{()}
\end{Highlighting}
\end{Shaded}

\includegraphics{STA380_James_Problems_files/figure-latex/prob 5 tSNE-1.pdf}

We than ran K means on TSNE and calculate the confusion matrix. It seems
to do a good with the reds, but there quite a few misclassified whites!
At first glance it looks like PCA does a much better job with accuracy,
as the calculated tSNE accruacy is 5.44\%

\begin{Shaded}
\begin{Highlighting}[]
\NormalTok{K }\OtherTok{=} \DecValTok{2}
\NormalTok{kmeansTSNEResultColor }\OtherTok{=} \FunctionTok{kmeans}\NormalTok{(tsneData[, }\DecValTok{1}\SpecialCharTok{:}\DecValTok{2}\NormalTok{], }\AttributeTok{centers =}\NormalTok{ K, }\AttributeTok{nstart =} \DecValTok{20}\NormalTok{)}
\NormalTok{tsneData}\SpecialCharTok{$}\NormalTok{clusterColor }\OtherTok{=} \FunctionTok{as.factor}\NormalTok{(kmeansTSNEResultColor}\SpecialCharTok{$}\NormalTok{cluster)}

\CommentTok{\# Create a confusion matrix}
\NormalTok{confusion\_matrix\_tsne }\OtherTok{\textless{}{-}} \FunctionTok{table}\NormalTok{(tsneData}\SpecialCharTok{$}\NormalTok{color, tsneData}\SpecialCharTok{$}\NormalTok{clusterColor)}
\FunctionTok{print}\NormalTok{(confusion\_matrix\_tsne)}
\end{Highlighting}
\end{Shaded}

\begin{verbatim}
##        
##            1    2
##   red     13 1344
##   white 3577  382
\end{verbatim}

\begin{Shaded}
\begin{Highlighting}[]
\CommentTok{\# Calculate the accuracy}
\NormalTok{correct\_labels\_tsne }\OtherTok{\textless{}{-}} \FunctionTok{sum}\NormalTok{(}\FunctionTok{diag}\NormalTok{(confusion\_matrix\_tsne))  }\CommentTok{\# Sum of the diagonal elements}
\NormalTok{total\_labels\_tsne }\OtherTok{\textless{}{-}} \FunctionTok{sum}\NormalTok{(confusion\_matrix\_tsne)  }\CommentTok{\# Sum of all elements}
\NormalTok{accuracy\_tsne }\OtherTok{\textless{}{-}}\NormalTok{ correct\_labels\_tsne }\SpecialCharTok{/}\NormalTok{ total\_labels\_tsne  }\CommentTok{\# Accuracy}

\CommentTok{\# Print the accuracy}
\FunctionTok{print}\NormalTok{(}\FunctionTok{paste}\NormalTok{(}\StringTok{"Accuracy: "}\NormalTok{, }\FunctionTok{round}\NormalTok{(accuracy\_tsne, }\DecValTok{4}\NormalTok{)))}
\end{Highlighting}
\end{Shaded}

\begin{verbatim}
## [1] "Accuracy:  0.0743"
\end{verbatim}

\begin{Shaded}
\begin{Highlighting}[]
\CommentTok{\# Visualize t{-}SNE with K{-}means clusters}
\FunctionTok{ggplot}\NormalTok{(tsneData, }\FunctionTok{aes}\NormalTok{(}\AttributeTok{x =}\NormalTok{ TSNE\_1, }\AttributeTok{y =}\NormalTok{ TSNE\_2, }\AttributeTok{color =}\NormalTok{ clusterColor)) }\SpecialCharTok{+}
  \FunctionTok{geom\_point}\NormalTok{() }\SpecialCharTok{+}
  \FunctionTok{labs}\NormalTok{(}\AttributeTok{title =} \StringTok{"t{-}SNE Visualization with Clusters for Color"}\NormalTok{, }\AttributeTok{x =} \StringTok{"t{-}SNE Dimension 1"}\NormalTok{, }\AttributeTok{y =} \StringTok{"t{-}SNE Dimension 2"}\NormalTok{) }\SpecialCharTok{+}
  \FunctionTok{scale\_color\_manual}\NormalTok{(}\AttributeTok{values =} \FunctionTok{c}\NormalTok{(}\StringTok{"2"} \OtherTok{=} \StringTok{"blue"}\NormalTok{, }\StringTok{"1"} \OtherTok{=} \StringTok{"red"}\NormalTok{))}\SpecialCharTok{+}
  \FunctionTok{theme\_minimal}\NormalTok{()}
\end{Highlighting}
\end{Shaded}

\includegraphics{STA380_James_Problems_files/figure-latex/problem 4 k means on tsne-1.pdf}

TSNE may do a little better job are predicting quality, but it still
doesn't look great. The plot below is quite a mess, with the quality all
over the place in two clusters. The boundary at the top of the second
cluster is quite interesting. It looks like most predicted wine quality
is 5 or 6.

\begin{Shaded}
\begin{Highlighting}[]
\NormalTok{tsneDataQuality }\OtherTok{=} \FunctionTok{data.frame}\NormalTok{(}
  \AttributeTok{TSNE\_1 =}\NormalTok{ tsneResult}\SpecialCharTok{$}\NormalTok{Y[, }\DecValTok{1}\NormalTok{], }
  \AttributeTok{TSNE\_2 =}\NormalTok{ tsneResult}\SpecialCharTok{$}\NormalTok{Y[, }\DecValTok{2}\NormalTok{], }
  \AttributeTok{quality =}\NormalTok{ wine}\SpecialCharTok{$}\NormalTok{quality, }
  \AttributeTok{cluster =}\NormalTok{ tsneData}\SpecialCharTok{$}\NormalTok{clusterColor}
\NormalTok{)}

\FunctionTok{ggplot}\NormalTok{(tsneDataQuality, }\FunctionTok{aes}\NormalTok{(}\AttributeTok{x =}\NormalTok{ TSNE\_1, }\AttributeTok{y =}\NormalTok{ TSNE\_2, }\AttributeTok{color =} \FunctionTok{as.factor}\NormalTok{(quality))) }\SpecialCharTok{+}
  \FunctionTok{geom\_point}\NormalTok{() }\SpecialCharTok{+}
  \FunctionTok{facet\_wrap}\NormalTok{(}\SpecialCharTok{\textasciitilde{}}\NormalTok{ cluster) }\SpecialCharTok{+}
  \FunctionTok{labs}\NormalTok{(}\AttributeTok{title =} \StringTok{"t{-}SNE Visualization by Quality and Cluster"}\NormalTok{, }
       \AttributeTok{x =} \StringTok{"t{-}SNE Dimension 1"}\NormalTok{, }\AttributeTok{y =} \StringTok{"t{-}SNE Dimension 2"}\NormalTok{) }\SpecialCharTok{+}
  \FunctionTok{theme\_minimal}\NormalTok{()}
\end{Highlighting}
\end{Shaded}

\includegraphics{STA380_James_Problems_files/figure-latex/prob 5 TSNE quality-1.pdf}
Overall, I would pick PCA for clustering the wine into reds or whites.
Even though PCA is linear in nature, it does a better job at clustering.
Predicting the color is objective computed to wine quality which is
subjective. It is cool that these unsupervised techniques seem to be
doing a pretty good job on clustering with very few parameters.

\end{document}
